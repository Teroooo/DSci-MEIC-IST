%\documentclass[10pt,draft]{extarticle}
\documentclass[10pt]{extarticle}
\renewcommand{\arraystretch}{1.5}
\renewcommand{\baselinestretch}{1.5}
\usepackage[onehalfspacing]{setspace}
\setlength{\parindent}{0em}
\setlength{\parskip}{0.2em}
\font\myfont=cmr12 at 26pt
\usepackage{anyfontsize}
\usepackage{tabularx}
\usepackage{multirow}
\pagenumbering{arabic} 
\usepackage{soul}
\usepackage{xcolor}
\usepackage[T1]{fontenc}
\renewcommand{\familydefault}{\sfdefault}
\usepackage{blindtext}
\usepackage{titling}
\setlength{\droptitle}{-14em}   % This is your set screw
\usepackage[english]{babel}
\usepackage{graphicx}
\usepackage{float}
\usepackage{eso-pic}
\graphicspath{ {./images/} }
\usepackage{subcaption}
\usepackage{geometry}
\usepackage[section]{placeins}
\usepackage{hyperref}
\geometry{margin=2cm, bmargin=2cm, tmargin=3cm}

\newcommand{\ctext}[3][RGB]{%
  \begingroup
  \definecolor{hlcolor}{#1}{#2}\sethlcolor{hlcolor}%
  \hl{#3}%
  \endgroup
}


\begin{document}


\AddToShipoutPictureBG*{
\AtPageUpperLeft{
\hspace{19.5cm}
\raisebox{-2.5cm}{\makebox[0pt][r]{\fontsize{36}{1cm}\selectfont DS 2025\par}}}}

\AddToShipoutPictureBG*{
\AtPageUpperLeft{
\hspace{6.5cm}
\raisebox{-3.5cm}{
\makebox[0pt][r]{ 
\includegraphics[scale=0.95]{tecnico_logo.jpg}\\[3cm]}}}}


\title{{\myfont Data Science Project}}  % Title
\setlength{\droptitle}{1cm}

\date{\vspace{-9ex}} % Date for the report, skipped and used to adjust height
\maketitle % Insert the title, author and date
\begin{center}
    %\setlength\extrarowheight{7pt}
    \begin{tabular}{ |l|l l|l| }
        \hline
        \multirow{4}{6em}{\textbf{Team nr: } 16} & \textbf{Student 1: } Antero Morgado & \textbf{IST nr: } 1119213\\
        & \textbf{Student 2: } David Ferreira & \textbf{IST nr: } 1107077 \\
        & \textbf{Student 3: } José Fernandes & \textbf{IST nr: } 1103727 \\
        & \textbf{Student 4: } Olha Buts & \textbf{IST nr: } 1116276 \\
        \hline
    \end{tabular}
\end{center}

\begin{center}
	\section*{\fontsize{0.75cm}{1cm}\selectfont CLASSIFICATION}
\end{center}

\section{DATA PROFILING}
\ctext[RGB]{190,190,190}{May be used to describe any useful observation about the data, and that was used in the current project. An example is the use of any domain knowledge to process the data or evaluate the results. \textbf{Shall not exceed 200 characters.}}

\subsection*{\textit{Data Dimensionality}}

Regarding the Missing Values analysis, the completeness metric for Dataset 1 was adjusted to treat 'UNKNOWN' occurrences as missing values.

\begin{figure}[H]
\centering
\begin{subfigure}{0.48\textwidth}
\centering\includegraphics[width=\textwidth]{traffic_accidents/dimensionality/traffic_accidents_records_variables.png}
\end{subfigure}
\hfill
\begin{subfigure}{0.48\textwidth}
\centering\includegraphics[width=\textwidth]{Combined_Flights_2022/dimensionality/Combined_Flights_2022_records_variables.png}
\end{subfigure}
\caption{Nr Records x Nr variables for dataset 1 (left) and dataset 2 (right)}
\end{figure}

\begin{figure}[H]
\centering
\begin{subfigure}{0.48\textwidth}
\centering\includegraphics[width=\textwidth]{traffic_accidents/dimensionality/traffic_accidents_variable_types.png}
\end{subfigure}
\hfill
\begin{subfigure}{0.48\textwidth}
\centering\includegraphics[width=\textwidth]{Combined_Flights_2022/dimensionality/Combined_Flights_2022_variable_types.png}
\end{subfigure}
\caption{Nr variables per type for dataset 1 (left) and dataset 2 (right)}
\end{figure}

\begin{figure}[H]
\centering
\begin{subfigure}{0.48\textwidth}
\centering\includegraphics[width=\textwidth]{traffic_accidents/dimensionality/traffic_accidents_mv.png}
\end{subfigure}
\hfill
\begin{subfigure}{0.48\textwidth}
\centering\includegraphics[width=\textwidth]{Combined_Flights_2022/dimensionality/Combined_Flights_2022_mv.png}
\end{subfigure}
\caption{Nr missing values for dataset 1 (left) and dataset 2 (right)}
\end{figure}

\subsection*{\textit{Data Distribution}}

The distribution analysis reveals that the raw data is significantly dispersed, exhibiting high variance across several features. Certain variables show a clear lack of balance, with distributions heavily skewed toward specific classes or ranges and presence of evident outliers.

\begin{figure}[H]
\centering
\begin{subfigure}{0.48\textwidth}
\centering\includegraphics[width=\textwidth]{traffic_accidents/distribution/traffic_accidents_global_boxplot.png}
\end{subfigure}
\hfill
\begin{subfigure}{0.48\textwidth}
\centering\includegraphics[width=\textwidth]{Combined_Flights_2022/distribution/Combined_Flights_2022_global_boxplot.png}
\end{subfigure}
\caption{Global boxplots dataset 1 (left) and dataset 2 (right)}
\end{figure}

\begin{figure}[H]
  \centering\includegraphics[width=0.9\textwidth]{traffic_accidents/distribution/traffic_accidents_single_boxplots.png}
  \caption{Single variables boxplots for dataset 1}
\end{figure}

\begin{figure}[H]
  \centering\includegraphics[width=0.3\textwidth]{Combined_Flights_2022/distribution/Combined_Flights_2022_single_boxplots.png}
  \caption{Single variables boxplots for dataset 2}
\end{figure}

\begin{figure}[H]
  \centering\includegraphics[width=0.7\textwidth]{traffic_accidents/distribution/traffic_accidents_single_histograms_numeric.png}
  \caption{Histograms for dataset 1} %(with distributions is enough)
\end{figure}

\begin{figure}[H]
  \centering\includegraphics[width=0.3\textwidth]{Combined_Flights_2022/distribution/Combined_Flights_2022_single_histograms_numeric.png}
  \caption{Histograms for dataset 2} %(with distributions is enough)
\end{figure}

\begin{figure}[H]
  \centering\includegraphics[width=0.9\textwidth]{traffic_accidents/distribution/traffic_accidents_outliers_standard.png}
  \caption{Outliers study dataset 1}
\end{figure}

\begin{figure}[H]
  \centering\includegraphics[width=0.9\textwidth]{Combined_Flights_2022/distribution/Combined_Flights_2022_outliers_standard.png}
  \caption{Outliers study dataset 2}
\end{figure}

\begin{figure}[H]
  \centering\includegraphics[width=0.9\textwidth]{traffic_accidents/distribution/traffic_accidents_class_distribution.png}
  \caption{Class distribution for dataset 1}
\end{figure}

\begin{figure}[H]
  \centering\includegraphics[width=0.5\textwidth]{Combined_Flights_2022/distribution/Combined_Flights_2022_histograms_symbolic.png}
  \caption{Class distribution for dataset 2}
\end{figure}

\subsection*{\textit{Data Granularity}}

The granularity analysis highlights significant variability across variables in both datasets. Categorical features such as weather, lighting conditions, airline, and airport present low to medium granularity, while temporal and numerical variables (hour, delays, distance) show higher granularity and dispersion. These differences suggest the need for appropriate grouping or discretization strategies to avoid sparsity and improve downstream analysis and modeling.

\begin{figure}[H]
\centering
\begin{subfigure}{0.32\textwidth}
\centering\includegraphics[width=\textwidth]{traffic_accidents/granularity/traffic_accidents_granularity_weather.png}
\caption{Weather condition}
\end{subfigure}
\hfill
\begin{subfigure}{0.32\textwidth}
\centering\includegraphics[width=\textwidth]{traffic_accidents/granularity/traffic_accidents_granularity_day.png}
\caption{Day of week}
\end{subfigure}
\hfill
\begin{subfigure}{0.32\textwidth}
\centering\includegraphics[width=\textwidth]{traffic_accidents/granularity/traffic_accidents_granularity_hour.png}
\caption{Hour of day}
\end{subfigure}

\vspace{0.2cm}

\begin{subfigure}{0.32\textwidth}
\centering\includegraphics[width=\textwidth]{traffic_accidents/granularity/traffic_accidents_granularity_month.png}
\caption{Month}
\end{subfigure}
\hfill
\begin{subfigure}{0.32\textwidth}
\centering\includegraphics[width=\textwidth]{traffic_accidents/granularity/traffic_accidents_granularity_lighting.png}
\caption{Lighting condition}
\end{subfigure}
\hfill
\begin{subfigure}{0.32\textwidth}
\centering\includegraphics[width=\textwidth]{traffic_accidents/granularity/traffic_accidents_granularity_surface.png}
\caption{Roadway surface}
\end{subfigure}

\vspace{0.2cm}

\begin{subfigure}{0.32\textwidth}
\centering\includegraphics[width=\textwidth]{traffic_accidents/granularity/traffic_accidents_granularity_trafficway.png}
\caption{Trafficway type}
\end{subfigure}
\hfill
\begin{subfigure}{0.32\textwidth}
\centering\includegraphics[width=\textwidth]{traffic_accidents/granularity/traffic_accidents_granularity_control.png}
\caption{Traffic control}
\end{subfigure}
\hfill
\begin{subfigure}{0.32\textwidth}
\centering\includegraphics[width=\textwidth]{traffic_accidents/granularity/traffic_accidents_granularity_crash_type.png}
\caption{Crash type}
\end{subfigure}

\vspace{0.2cm}

\begin{subfigure}{0.32\textwidth}
\centering\includegraphics[width=\textwidth]{traffic_accidents/granularity/traffic_accidents_granularity_damage.png}
\caption{Damage level}
\end{subfigure}
\hfill
\begin{subfigure}{0.32\textwidth}
\centering\includegraphics[width=\textwidth]{traffic_accidents/granularity/traffic_accidents_granularity_injury.png}
\caption{Injury severity}
\end{subfigure}
\hfill
\begin{subfigure}{0.32\textwidth}
\centering\includegraphics[width=\textwidth]{traffic_accidents/granularity/traffic_accidents_granularity_num_units.png}
\caption{Number of units}
\end{subfigure}

\caption{Granularity analysis for dataset 1}
\end{figure}


\begin{figure}[H]
\centering
\begin{subfigure}{0.32\textwidth}
\centering\includegraphics[width=\textwidth]{Combined_Flights_2022/granularity/Combined_Flights_2022_granularity_airline.png}
\caption{Airline}
\end{subfigure}
\hfill
\begin{subfigure}{0.32\textwidth}
\centering\includegraphics[width=\textwidth]{Combined_Flights_2022/granularity/Combined_Flights_2022_granularity_origin.png}
\caption{Origin airport}
\end{subfigure}
\hfill
\begin{subfigure}{0.32\textwidth}
\centering\includegraphics[width=\textwidth]{Combined_Flights_2022/granularity/Combined_Flights_2022_granularity_destination.png}
\caption{Destination airport}
\end{subfigure}

\vspace{0.2cm}

\begin{subfigure}{0.32\textwidth}
\centering\includegraphics[width=\textwidth]{Combined_Flights_2022/granularity/Combined_Flights_2022_granularity_day.png}
\caption{Day of week}
\end{subfigure}
\hfill
\begin{subfigure}{0.32\textwidth}
\centering\includegraphics[width=\textwidth]{Combined_Flights_2022/granularity/Combined_Flights_2022_granularity_month.png}
\caption{Month}
\end{subfigure}
\hfill
\begin{subfigure}{0.32\textwidth}
\centering\includegraphics[width=\textwidth]{Combined_Flights_2022/granularity/Combined_Flights_2022_granularity_dayofmonth.png}
\caption{Day of month}
\end{subfigure}

\vspace{0.2cm}

\begin{subfigure}{0.32\textwidth}
\centering\includegraphics[width=\textwidth]{Combined_Flights_2022/granularity/Combined_Flights_2022_granularity_distance.png}
\caption{Distance}
\end{subfigure}
\hfill
\begin{subfigure}{0.32\textwidth}
\centering\includegraphics[width=\textwidth]{Combined_Flights_2022/granularity/Combined_Flights_2022_granularity_airtime.png}
\caption{Air time}
\end{subfigure}
\hfill
\begin{subfigure}{0.32\textwidth}
\centering\includegraphics[width=\textwidth]{Combined_Flights_2022/granularity/Combined_Flights_2022_granularity_dep_delay.png}
\caption{Departure delay}
\end{subfigure}

\vspace{0.2cm}

\begin{subfigure}{0.32\textwidth}
\centering\includegraphics[width=\textwidth]{Combined_Flights_2022/granularity/Combined_Flights_2022_granularity_arr_delay.png}
\caption{Arrival delay}
\end{subfigure}

\caption{Granularity analysis for dataset 2}
\end{figure}


\subsection*{\textit{Data Sparsity}}

The sparsity analysis reveals uneven domain coverage across variables, with several categorical features dominated by a small number of frequent values and many rare ones. Correlation analysis shows limited strong dependencies between most variables, indicating sparse relationships in the feature space.

\begin{figure}[H]
  \centering\includegraphics[width=0.9\textwidth]{traffic_accidents/sparsity/traffic_accidents_sparsity_per_class_study.png}
  \caption{Sparsity analysis for dataset 1}
\end{figure}

\begin{figure}[H]
%\centering\includegraphics[width=0.9\textwidth]{Combined_Flights_2022/sparsity/FILENAME.png}
\caption{Sparsity analysis for dataset 2 - \href{https://drive.google.com/file/d/1fxiYqSNrlIkwIcOJLn5kSXHX5Qk8-tKo/view?usp=sharing}{View on Google Drive}}
\end{figure}

\begin{figure}[H]
  \centering\includegraphics[width=0.9\textwidth]{traffic_accidents/dimensionality/traffic_accidents_correlation_analysis.png}
  \caption{Correlation analysis for dataset 1}
\end{figure}

\begin{figure}[H]
  \centering\includegraphics[width=0.9\textwidth]{Combined_Flights_2022/sparsity/Combined_Flights_2022_correlation_analysis.png}
  \caption{Correlation analysis for dataset 2}
\end{figure}

\section{DATA PREPARATION}

\subsection*{\textit{Variables Encoding}}
\ctext[RGB]{190,190,190}{Shall contain all relevant information respecting to the transformation of variables. The list of variables under each one of the transformations, shall be presented. If not applied explain the reason for that, based on data characteristics.  \textbf{Shall not exceed 500 characters \textit{for each dataset.}}}

\subsection*{\textit{Missing Value Imputation}}
\ctext[RGB]{190,190,190}{Shall contain all relevant information and charts respecting to missing values imputation, such as the choices made and the impact of the different approaches on modelling results. Shall also clearly reveal the approach selected to proceed with the processing. If not applied explain the reason for that, based on data characteristics.  \textbf{Shall not exceed 500 characters.}}

\begin{figure}[H]
\centering\includegraphics[scale=0.95]{images/}
\caption{Missing values imputation results with different approaches for dataset 1}
\end{figure}

\begin{figure}[H]
%\centering\includegraphics[scale=0.95]{}
\caption{Missing values imputation results with different approaches for dataset 2}
\end{figure}

\subsection*{\textit{Outliers Treatment}}
\ctext[RGB]{190,190,190}{Shall contain all relevant information and charts respecting to outliers imputation, such as the choices made and the impact of the different approaches on modelling results. Shall also clearly reveal the approach selected to proceed with the processing. If not applied explain the reason for that, based on data characteristics.  \textbf{Shall not exceed 500 characters.}}

\begin{figure}[H]
%\centering\includegraphics[scale=0.95]{}
\caption{Outliers imputation results with different approaches for dataset 1}
\end{figure}

\begin{figure}[H]
%\centering\includegraphics[scale=0.95]{}
\caption{Outliers imputation results with different approaches for dataset 2}
\end{figure}

\subsection*{\textit{Scaling}}
\ctext[RGB]{190,190,190}{Shall contain all relevant information and charts respecting to scaling transformation, such as the choices made and the impact of the different approaches on modelling results. Shall also clearly reveal the approach selected to proceed with the processing. If not applied explain the reason for that, based on data characteristics.  \textbf{Shall not exceed 200 characters.}}

\begin{figure}[H]
%\centering\includegraphics[scale=0.95]{}
\caption{Scaling results with different approaches for dataset 1}
\end{figure}

\begin{figure}[H]
%\centering\includegraphics[scale=0.95]{}
\caption{Scaling results with different approaches for dataset 2}
\end{figure}

\subsection*{\textit{Balancing}}
\ctext[RGB]{190,190,190}{Shall contain all relevant information and charts respecting to balancing transformation, such as the choices made and the impact of the different approaches on modelling results. Shall also clearly reveal the approach selected to proceed with the processing. If not applied explain the reason for that, based on data characteristics.  \textbf{Shall not exceed 500 characters.}}

\begin{figure}[H]
%\centering\includegraphics[scale=0.95]{}
\caption{Balancing results with different approaches for dataset 1}
\end{figure}

\begin{figure}[H]
%\centering\includegraphics[scale=0.95]{}
\caption{Balancing results with different approaches for dataset 2}
\end{figure}

\subsection*{\textit{Feature Selection}}
\ctext[RGB]{190,190,190}{Shall contain all relevant information and charts respecting to feature selection based on filtering out redundant (based on correlation) and relevant (based on variation) variables. The different choices and their impact on the modelling results shall be presented and explained. Should also clearly reveal the approach selected to proceed with the processing. All explanations shall be based on data characteristics.  \textbf{Shall not exceed 500 characters.}}

\begin{figure}[H]
%\centering\includegraphics[scale=0.95]{}
\caption{Feature selection of redundant variables results with different parameters for dataset 1}
\end{figure}

\begin{figure}[H]
%\centering\includegraphics[scale=0.95]{}
\caption{Feature selection of redundant variables results with different parameters for dataset 2}
\end{figure}

\begin{figure}[H]
%\centering\includegraphics[scale=0.95]{}
\caption{Feature selection of relevant variables results with different parameters for dataset 1 (variance study)} 
\end{figure}

\begin{figure}[H]
%\centering\includegraphics[scale=0.95]{}
\caption{Feature selection of relevant variables results with different parameters for dataset 2 (variance study)}
\end{figure}

\subsection*{\textit{Feature Extraction (optional)}}
\ctext[RGB]{190,190,190}{Shall contain all relevant information and charts respecting to feature extraction, in particular PCA. The different choices and their impact on the modelling results shall be presented and explained.  \textbf{Shall not exceed 200 characters.}}

\begin{figure}[H]
%\centering\includegraphics[scale=0.95]{}
\caption{Principal components analysis and feature extraction results for dataset 1}
\end{figure}

\begin{figure}[H]
%\centering\includegraphics[scale=0.95]{}
\caption{Principal components analysis and feature extraction results for dataset 2}
\end{figure}


\section{MODELS' EVALUATION}
\ctext[RGB]{190,190,190}{Shall be used to point out any important decision taken during the training, including training strategy and evaluation measures used.  \textbf{Shall not exceed 500 characters.}}

\subsection*{\textit{Na{\"i}ve Bayes}}
\ctext[RGB]{190,190,190}{Shall be used to present the results achieved with each one of Na{\"i}ve Bayes implementations, comparing and proposing explanations for them. If any of the implementations is not used, a justification for it shall be presented. Shall be used to present the evaluation of the best model achieved.  \textbf{Shall not exceed 300 characters.}}

\begin{figure}[H]
\centering\includegraphics[scale=0.5]{traffic_accidents/models_eval/NB/traffic accidents_nb_accuracy_study.png}
\caption{Na{\"i}ve Bayes alternatives comparison for dataset 1}
\end{figure}

\begin{figure}[H]
\centering\includegraphics[scale=0.5]{Combined_Flights_2022/models_eval/NB/Combined_flight_v1_nb_accuracy_study.png}
\caption{Na{\"i}ve Bayes alternative comparison for dataset 2}
\end{figure}

\begin{figure}[H]
\centering
\begin{subfigure}{0.45\textwidth}
\includegraphics[width=\textwidth]{traffic_accidents/models_eval/NB/traffic accidents_BernoulliNB_best_recall_eval.png}
\end{subfigure}
\hfill
\begin{subfigure}{0.45\textwidth}
\includegraphics[width=\textwidth]{Combined_Flights_2022/models_eval/NB/Combined_flight_v1_BernoulliNB_best_recall_eval.png}
\end{subfigure}
\caption{Na{\"i}ve Bayes best model results for dataset 1 (left) and dataset 2 (right)}
\end{figure}

\subsection*{\textit{KNN}}
\ctext[RGB]{190,190,190}{Shall be used to present the results achieved through different similarity measures and KNN parameterisations. The results shall be compared and explanations for them shall be presented. The justification for the chosen similarity measures shall be presented. Shall be used to address the overfitting phenomenon, studying the conditions under which models face it. Shall be used to present the evaluation of the best model achieved.  \textbf{Shall not exceed 500 characters.}}

\begin{figure}[H]
\centering\includegraphics[scale=0.5]{traffic_accidents/models_eval/KNN/traffic_knn_accuracy_study.png}
\caption{KNN different parameterisations comparison for dataset 1}
\end{figure}

\begin{figure}[H]
\centering\includegraphics[scale=0.5]{Combined_Flights_2022/models_eval/KNN/cflights_knn_accuracy_study.png}
\caption{KNN different parameterisations comparison for dataset 2}
\end{figure}

\begin{figure}[H]
\centering
\begin{subfigure}{0.45\textwidth}
\includegraphics[width=\textwidth]{traffic_accidents/models_eval/KNN/traffic_knn_overfitting.png}
\end{subfigure}
\hfill
\begin{subfigure}{0.45\textwidth}
\includegraphics[width=\textwidth]{Combined_Flights_2022/models_eval/KNN/cflights_knn_overfitting.png}
\end{subfigure}
\caption{KNN overfitting analysis for dataset 1 (left) and dataset 2 (right)}
\end{figure}

\begin{figure}[H]
\centering
\begin{subfigure}{0.45\textwidth}
\includegraphics[width=\textwidth]{traffic_accidents/models_eval/KNN/traffic_knn_KNN_best_accuracy_eval.png}
\end{subfigure}
\hfill
\begin{subfigure}{0.45\textwidth}
\includegraphics[width=\textwidth]{Combined_Flights_2022/models_eval/KNN/cflights_knn_KNN_best_accuracy_eval.png}
\end{subfigure}
\caption{KNN best model results for dataset 1 (left) and dataset 2 (right)}
\end{figure}


\subsection*{\textit{Logistic Regression}}

\begin{figure}[H]
\centering\includegraphics[scale=0.5]{traffic_accidents/models_eval/LR/traffic_lr_accuracy_study.png}
\caption{Logistic Regression different parameterisations comparison for dataset 1}
\end{figure}

\begin{figure}[H]
\centering\includegraphics[scale=0.5]{Combined_Flights_2022/models_eval/LR/cflights_lr_accuracy_study.png}
\caption{Logistic Regression different parameterisations comparison for dataset 2}
\end{figure}

\begin{figure}[H]
\centering
\begin{subfigure}{0.45\textwidth}
\includegraphics[width=\textwidth]{traffic_accidents/models_eval/LR/traffic_lr_accuracy_overfitting.png}
\end{subfigure}
\hfill
\begin{subfigure}{0.45\textwidth}
\includegraphics[width=\textwidth]{Combined_Flights_2022/models_eval/LR/cflights_lr_accuracy_overfitting.png}
\end{subfigure}
\caption{Logistic Regression overfitting analysis for dataset 1 (left) and dataset 2 (right)}
\end{figure}

\begin{figure}[H]
\centering
\begin{subfigure}{0.45\textwidth}
\includegraphics[width=\textwidth]{traffic_accidents/models_eval/LR/traffic_lr_LR_best_accuracy_eval.png}
\end{subfigure}
\hfill
\begin{subfigure}{0.45\textwidth}
\includegraphics[width=\textwidth]{Combined_Flights_2022/models_eval/LR/cflights_lr_LR_best_accuracy_eval.png}
\end{subfigure}
\caption{Logistic Regression best model results for dataset 1 (left) and dataset 2 (right)}
\end{figure}

\begin{figure}[H]
\centering\includegraphics[scale=0.5]{traffic_accidents/models_eval/LR/feature_importance1.jpeg}
\caption{Logistic Regression feature importance for dataset 1}
\end{figure}

\begin{figure}[H]
\centering\includegraphics[scale=0.5]{Combined_Flights_2022/models_eval/LR/feature_importance2.jpeg}
\caption{Logistic Regression feature importance for dataset 2}
\end{figure}


\subsection*{\textit{Decision Trees}}
Decision trees behave differently across the two datasets.In Dataset~1, accuracy peaks around depth~6--8 (approximately 0.72) and then declines slightly.The best model (entropy, depth~6) reaches 0.74 accuracy with balanced precision and recall; key features include \textit{crash\_type}, \textit{damage}, \textit{dry\_cause}, \textit{num\_units}, and \textit{ash\_date}. In Dataset~2, entropy performs better, although deep trees overfit.The best model (depth~24) achieves 0.87 accuracy.
\begin{figure}[H]
\centering\includegraphics[scale=0.5]{traffic_accidents/models_eval/DT/Traffic_dt_accuracy_study.png}
\caption{Decision Trees different parameterisations comparison for dataset 1}
\end{figure}

\begin{figure}[H]
\centering\includegraphics[scale=0.5]{Combined_Flights_2022/models_eval/DT/Flight_dt_accuracy_study.png}
\caption{Decision Trees different parameterisations comparison for dataset 2}
\end{figure}

\begin{figure}[H]
\centering
\begin{subfigure}{0.45\textwidth}
\includegraphics[width=\textwidth]{traffic_accidents/models_eval/DT/Traffic_dt_overfitting.png}
\end{subfigure}
\hfill
\begin{subfigure}{0.45\textwidth}
\includegraphics[width=\textwidth]{Combined_Flights_2022/models_eval/DT/Flight_dt_accuracy_overfitting.png}
\end{subfigure}
\caption{Decision Trees overfitting analysis for dataset 1 (left) and dataset 2 (right)}
\end{figure}

\begin{figure}[H]
\centering
\begin{subfigure}{0.45\textwidth}
\includegraphics[width=\textwidth]{traffic_accidents/models_eval/DT/Traffic_dt_DT_best_accuracy_eval.png}
\end{subfigure}
\hfill
\begin{subfigure}{0.45\textwidth}
\includegraphics[width=\textwidth]{Combined_Flights_2022/models_eval/DT/Flight_dt_DT_best_accuracy_eval.png}
\end{subfigure}
\caption{Decision trees best model results for dataset 1 (left) and dataset 2 (right)}
\end{figure}

\begin{figure}[H]
\centering\includegraphics[width=0.9\textwidth]{traffic_accidents/models_eval/DT/Traffic_dt_accuracy_best_tree.png}
\caption{Best tree for dataset 1}
\end{figure}

\begin{figure}[H]
\centering\includegraphics[width=0.7\textwidth]{Combined_Flights_2022/models_eval/DT/Flight_dt_accuracy_best_tree_V2.png}
\caption{Best tree for dataset 2}
\end{figure}

\subsection*{\textit{Random Forests}}
Random forests improve steadily in both datasets.In Dataset~1, the best model (depth~5, feature fraction~0.5, 750 trees) reaches 0.72 test accuracy with balanced metrics.In Dataset~2, accuracy increases quickly with more trees and moderate depth.The best configuration (depth~7, feature fraction~0.3, 750 trees) achieves 0.79 accuracy.
\begin{figure}[H]
\centering\includegraphics[scale=0.5]{traffic_accidents/models_eval/RF/Traffic_rf_accuracy_study.png}
\caption{Random Forests different parameterisations comparison for dataset 1}
\end{figure}

\begin{figure}[H]
\centering\includegraphics[scale=0.5]{Combined_Flights_2022/models_eval/RF/Flight_rf_accuracy_study.png}
\caption{Random Forests different parameterisations comparison for dataset 2}
\end{figure}

\begin{figure}[H]
\centering
\begin{subfigure}{0.45\textwidth}
\includegraphics[width=\textwidth]{traffic_accidents/models_eval/RF/Traffic_rf_accuracy_overfitting.png}
\end{subfigure}
\hfill
\begin{subfigure}{0.45\textwidth}
\includegraphics[width=\textwidth]{Combined_Flights_2022/models_eval/RF/Flight_rf_accuracy_overfitting.png}
\end{subfigure}
\caption{Random Forests overfitting analysis for dataset 1 (left) and dataset 2 (right)}
\end{figure}

\begin{figure}[H]
\centering
\begin{subfigure}{0.45\textwidth}
\includegraphics[width=\textwidth]{traffic_accidents/models_eval/RF/Traffic_rf_RF_best_accuracy_eval.png}
\end{subfigure}
\hfill
\begin{subfigure}{0.45\textwidth}
\includegraphics[width=\textwidth]{Combined_Flights_2022/models_eval/RF/Flight_rf_RF_best_accuracy_eval.png}
\end{subfigure}
\caption{Random Forests best model results for dataset 1 (left) and dataset 2 (right)}
\end{figure}

\begin{figure}[H]
\centering
\begin{subfigure}{0.45\textwidth}
\includegraphics[width=\textwidth]{traffic_accidents/models_eval/RF/Traffic_rf_accuracy_vars_ranking.png}
\end{subfigure}
\hfill
\begin{subfigure}{0.45\textwidth}
\includegraphics[width=\textwidth]{Combined_Flights_2022/models_eval/RF/Flight_rf_accuracy_vars_ranking.png}
\end{subfigure}
\caption{Random Forests variables importance for dataset 1 (left) and dataset 2 (right)}
\end{figure}

\subsection*{\textit{Gradient Boosting}}
\ctext[RGB]{190,190,190}{Shall be used to present the results achieved through different parameterisations for the train of gradient boosting. The results shall be compared and explanations for them shall be presented. Shall be used to address the overfitting phenomenon, studying the conditions under which models face it.  Shall be used to present the evaluation of the best model achieved. May be used to present the most important variables in the model.  \textbf{Shall not exceed 500 characters.}}

\begin{figure}[H]
\centering\includegraphics[scale=0.5]{traffic_accidents/models_eval/GB/OverfittingAnalysistreedepthimpact.png}
\caption{Gradient boosting different parameterisations comparison for dataset 1}
\end{figure}

\begin{figure}[H]
\centering\includegraphics[scale=0.5]{Combined_Flights_2022/models_eval/GB/impactoftreedepth.png}
\caption{Gradient boosting different parameterisations comparison for dataset 2}
\end{figure}

\begin{figure}[H]
\centering
\begin{subfigure}{0.45\textwidth}
\includegraphics[width=\textwidth]{traffic_accidents/models_eval/GB/OverfittingAnalysis.png}
\end{subfigure}
\hfill
\begin{subfigure}{0.45\textwidth}
\includegraphics[width=\textwidth]{Combined_Flights_2022/models_eval/GB/ImpactofNumberofTrees.png}
\end{subfigure}
\caption{Gradient boosting overfitting analysis for dataset 1 (left) and dataset 2 (right)}
\end{figure}

\begin{figure}[H]
\centering
\begin{subfigure}{0.45\textwidth}
\includegraphics[width=\textwidth]{traffic_accidents/models_eval/GB/Model_performance_metrics_traffic.png}
\end{subfigure}
\hfill
\begin{subfigure}{0.45\textwidth}
\includegraphics[width=\textwidth]{Combined_Flights_2022/models_eval/GB/modelperformancemetrics.png}
\end{subfigure}
\caption{Gradient boosting best model results for dataset 1 (left) and dataset 2 (right)}
\end{figure}

\begin{figure}[H]
\centering
\begin{subfigure}{0.45\textwidth}
\includegraphics[width=\textwidth]{traffic_accidents/models_eval/GB/confusion_matrix.png}
\end{subfigure}
\hfill
\begin{subfigure}{0.45\textwidth}
\includegraphics[width=\textwidth]{Combined_Flights_2022/models_eval/GB/confusion_matrix.png}
\end{subfigure}
\caption{Gradient boosting variables importance for dataset 1 (left) and dataset 2 (right)}
\end{figure}

\subsection*{\textit{Multi-Layer Perceptrons}}
\ctext[RGB]{190,190,190}{Shall be used to present the results achieved through different parameterisations for the train of MLPs. The results shall be compared and explanations for them shall be presented. Shall be used to address the overfitting phenomenon, studying the conditions under which models face it. Shall be used to present the evaluation of the best model achieved.  \textbf{Shall not exceed 500 characters.}}

\begin{figure}[H]
\centering\includegraphics[scale=0.5]{traffic_accidents/models_eval/MLP/traffic_accidents_mlp_accuracy_study.png}
\caption{MLP different parameterisations comparison for dataset 1}
\end{figure}

\begin{figure}[H]
\centering\includegraphics[scale=0.5]{Combined_Flights_2022/models_eval/MLP/Combined_flight_v1_mlp_accuracy_study.png}
\caption{MLP different parameterisations comparison for dataset 2}
\end{figure}

\begin{figure}[H]
\centering
\begin{subfigure}{0.45\textwidth}
\includegraphics[width=\textwidth]{traffic_accidents/models_eval/MLP/traffic_accidents_mlp_accuracy_overfitting.png}
\end{subfigure}
\hfill
\begin{subfigure}{0.45\textwidth}
\includegraphics[width=\textwidth]{Combined_Flights_2022/models_eval/MLP/Combined_flight_v1_mlp_accuracy_overfitting.png}
\end{subfigure}
\caption{MLP overfitting analysis for dataset 1 (left) and dataset 2 (right)}
\end{figure}

\begin{figure}[H]
\centering
\begin{subfigure}{0.45\textwidth}
\includegraphics[width=\textwidth]{traffic_accidents/models_eval/MLP/traffic_accidents_mlp_accuracy_loss_curve.png}
\end{subfigure}
\hfill
\begin{subfigure}{0.45\textwidth}
\includegraphics[width=\textwidth]{Combined_Flights_2022/models_eval/MLP/Combined_flight_v1_mlp_accuracy_loss_curve.png}
\end{subfigure}
\caption{Loss curve analysis for dataset 1 (left) and dataset 2 (right)}
\end{figure}

\begin{figure}[H]
\centering
\begin{subfigure}{0.45\textwidth}
\includegraphics[width=\textwidth]{traffic_accidents/models_eval/MLP/traffic_accidents_mlp_MLP_best_accuracy_eval.png}
\end{subfigure}
\hfill
\begin{subfigure}{0.45\textwidth}
\includegraphics[width=\textwidth]{Combined_Flights_2022/models_eval/MLP/Combined_flight_v1_mlp_MLP_best_accuracy_eval.png}
\end{subfigure}
\caption{MLP best model results for dataset 1 (left) and dataset 2 (right)}
\end{figure}

\section{CRITICAL ANALYSIS}
\ctext[RGB]{190,190,190}{Shall be used to present a summary of the results achieved with the different modelling techniques, and the impact of the different preparation tasks on their performance. 
A cross-analysis of the different models may also be presented, identifying the most relevant variables common to all of them (when possible) and the relation among the patterns identified within the different classifiers.
A critical assessment of the best models shall be presented, clearly stating if the models seem to be good enough for the problem at hand. \textbf{Additional charts may be presented here.  Shall not exceed 2000 characters.}}


\begin{center}
	\section*{\fontsize{0.75cm}{1cm}\selectfont TIME SERIES ANALYSIS}
\end{center}

\section{DATA PROFILING}

\subsection*{\textit{Data Dimensionality and Granularity}}

The time series was analyzed at three temporal granularities: 15-minute intervals (most atomic), daily aggregation, and weekly aggregation. The most granular level captures short-term variability and peaks, while daily and weekly granularities smooth fluctuations and highlight medium- and long-term trends.

\begin{figure}[H]
  \centering\includegraphics[width=0.7\textwidth]{TrafficTwoMonth/dimensionality/traffic_record.png}
  \caption{Time series 1 at the most granular detail}
\end{figure}

\begin{figure}[H]
\centering\includegraphics[width=0.8\textwidth]{TrafficTwoMonth/dimensionality/traffic_daily.png}
\caption{Time series 1 at the second chosen granularity}
\end{figure}

\begin{figure}[H]
\centering\includegraphics[width=0.8\textwidth]{TrafficTwoMonth/dimensionality/traffic_weekly.png}
\caption{Time series 1 at the third chosen granularity}
\end{figure}


\subsection*{\textit{Data Distribution}}

The distribution analysis across granularities shows that aggregation reduces variability and outliers. The 15-minute series presents higher dispersion and skewness, while hourly and daily series exhibit smoother distributions. 

\begin{figure}[H]
\centering\includegraphics[width=0.8\textwidth]{TrafficTwoMonth/data_distribution/5number_summary.png}
\caption{Boxplot(s) for time series 1}
\end{figure}

\begin{figure}[H]
\centering\includegraphics[width=0.8\textwidth]{TrafficTwoMonth/data_distribution/distributions.png}
\caption{Histogram(s) for time series 1}
\end{figure}

\begin{figure}[H]
\centering\includegraphics[width=0.8\textwidth]{TrafficTwoMonth/data_distribution/lagged_series.png}
\caption{Autocorrelation lag-plots for original time series 1}
\end{figure}

\begin{figure}[H]
\centering\includegraphics[width=0.8\textwidth]{TrafficTwoMonth/data_distribution/autocorrelation_study.png}
\caption{Autocorrelation correlogram for original time series 1}
\end{figure}

\subsection*{\textit{Data Stationarity}}

Stationarity analysis shows that the series is stationary at finer granularities but loses stationarity when aggregated daily. Augmented Dickey-Fuller Test confirm that trend components become more pronounced at coarser granularities, indicating the need for detrending or differencing when modeling aggregated series.

\begin{figure}[H]
\centering\includegraphics[width=0.8\textwidth]{TrafficTwoMonth/data_stationarity/components_daily.png}
\caption{Components study for time series 1}
\end{figure}


\begin{figure}[H]
\centering\includegraphics[width=0.8\textwidth]{TrafficTwoMonth/data_stationarity/stationarity_overall_mean.png}
\caption{Stationarity study for time series 1}
\end{figure}

\subsection*{\textit{Augmented Dickey-Fuller Test Results:}}

\textit{Original (15-min):}
\begin{itemize}
    \item ADF Statistic: -14.441
    \item p-value: 0.000
    \item Critical Values: 1\%: -3.431, 5\%: -2.862, 10\%: -2.567
    \item \textbf{The series IS stationary}
\end{itemize}

\textit{Hourly:}
\begin{itemize}
    \item ADF Statistic: -8.903
    \item p-value: 0.000
    \item Critical Values: 1\%: -3.435, 5\%: -2.864, 10\%: -2.568
    \item \textbf{The series IS stationary}
\end{itemize}

\textit{Daily:}
\begin{itemize}
    \item ADF Statistic: -0.826
    \item p-value: 0.811
    \item Critical Values: 1\%: -3.548, 5\%: -2.913, 10\%: -2.594
    \item \textbf{The series IS NOT stationary}
\end{itemize}


\section{DATA TRANSFORMATION}

\subsection*{\textit{Aggregation}}

We tested 30‑minute, hourly, daily, and weekly aggregations to see how each level reduces noise and how much detail is lost.The plots (see Figures \ref{fig:aggregation_plots}, \ref{fig:forecasting_plots}, and \ref{fig:evaluation_results}) reveal that no aggregation keeps high‑frequency noise, while daily and weekly levels smooth the data too much.On the test set, linear regression produced constant predictions with R² close to 0, signaling a poor fit to the data.Meanwhile, the optimistic persistence model performed better, and the 30‑minute level gave the best R² (~0.67), so we chose it.

\begin{figure}[H]
\centering
\begin{subfigure}[b]{0.3\textwidth}
    \centering
    \includegraphics[width=\textwidth]{images/TrafficTwoMonth/data_transformation/Scale/Traffic_after_scaling.png}
    \caption{No Aggregation (original after scaling)}
    \label{fig:agg:no}
\end{subfigure}
\hfill
\begin{subfigure}[b]{0.3\textwidth}
    \centering
    \includegraphics[width=\textwidth]{images/TrafficTwoMonth/data_transformation/Aggr/traffic_30min_aggregation.png}
    \caption{30min Aggregation}
    \label{fig:agg:30min}
\end{subfigure}
\hfill
\begin{subfigure}[b]{0.3\textwidth}
    \centering
    \includegraphics[width=\textwidth]{images/TrafficTwoMonth/data_transformation/Aggr/traffic_hourly_aggregation.png}
    \caption{Hourly Aggregation}
    \label{fig:agg:hourly}
\end{subfigure}

\vspace{1em}

\begin{subfigure}[b]{0.3\textwidth}
    \centering
    \includegraphics[width=\textwidth]{images/TrafficTwoMonth/data_transformation/Aggr/traffic_daily_aggregation.png}
    \caption{Daily Aggregation}
    \label{fig:agg:daily}
\end{subfigure}
\hfill
\begin{subfigure}[b]{0.3\textwidth}
    \centering
    \includegraphics[width=\textwidth]{images/TrafficTwoMonth/data_transformation/Aggr/traffic_weekly_aggregation.png}
    \caption{Weekly Aggregation}
    \label{fig:agg:weekly}
\end{subfigure}

\caption{Time series plots after different levels of aggregation}
\label{fig:aggregation_plots}
\end{figure}

\begin{figure}[H]
\centering
\begin{subfigure}[b]{0.45\textwidth}
    \centering
    \includegraphics[width=\textwidth]{images/TrafficTwoMonth/data_transformation/Aggr/traffic_linear_regression_forecast_No Aggregation.png}
    \caption{Linear Regression – No Aggregation}
\end{subfigure}
\hfill
\begin{subfigure}[b]{0.45\textwidth}
    \centering
    \includegraphics[width=\textwidth]{images/TrafficTwoMonth/data_transformation/Aggr/traffic_persistence_optim_forecast__No Aggregation.png}
    \caption{Persistence Optim – No Aggregation}
\end{subfigure}

\vspace{1em}

\begin{subfigure}[b]{0.45\textwidth}
    \centering
    \includegraphics[width=\textwidth]{images/TrafficTwoMonth/data_transformation/Aggr/traffic_linear_regression_forecast_30min Aggregation.png}
    \caption{Linear Regression – 30min Aggregation}
\end{subfigure}
\hfill
\begin{subfigure}[b]{0.45\textwidth}
    \centering
    \includegraphics[width=\textwidth]{images/TrafficTwoMonth/data_transformation/Aggr/traffic_persistence_optim_forecast__30min Aggregation.png}
    \caption{Persistence Optim – 30min Aggregation}
\end{subfigure}

\vspace{1em}

\begin{subfigure}[b]{0.45\textwidth}
    \centering
    \includegraphics[width=\textwidth]{images/TrafficTwoMonth/data_transformation/Aggr/traffic_linear_regression_forecast_Hourly Aggregation.png}
    \caption{Linear Regression – Hourly Aggregation}
\end{subfigure}
\hfill
\begin{subfigure}[b]{0.45\textwidth}
    \centering
    \includegraphics[width=\textwidth]{images/TrafficTwoMonth/data_transformation/Aggr/traffic_persistence_optim_forecast__Hourly Aggregation.png}
    \caption{Persistence Optim – Hourly Aggregation}
\end{subfigure}

\vspace{1em}

\begin{subfigure}[b]{0.45\textwidth}
    \centering
    \includegraphics[width=\textwidth]{images/TrafficTwoMonth/data_transformation/Aggr/traffic_linear_regression_forecast_Daily Aggregation.png}
    \caption{Linear Regression – Daily Aggregation}
\end{subfigure}
\hfill
\begin{subfigure}[b]{0.45\textwidth}
    \centering
    \includegraphics[width=\textwidth]{images/TrafficTwoMonth/data_transformation/Aggr/traffic_persistence_optim_forecast__Daily Aggregation.png}
    \caption{Persistence Optim – Daily Aggregation}
\end{subfigure}

\vspace{1em}

\begin{subfigure}[b]{0.45\textwidth}
    \centering
    \includegraphics[width=\textwidth]{images/TrafficTwoMonth/data_transformation/Aggr/traffic_linear_regression_forecast_Weekly Aggregation.png}
    \caption{Linear Regression – Weekly Aggregation}
\end{subfigure}
\hfill
\begin{subfigure}[b]{0.45\textwidth}
    \centering
    \includegraphics[width=\textwidth]{images/TrafficTwoMonth/data_transformation/Aggr/traffic_persistence_optim_forecast__Weekly Aggregation.png}
    \caption{Persistence Optim – Weekly Aggregation}
\end{subfigure}

\caption{Forecasting plots for Linear Regression and Persistence Optim after different aggregations}
\label{fig:forecasting_plots}
\end{figure}

\begin{figure}[H]
\centering
\begin{subfigure}[b]{0.40\textwidth}
    \centering
    \includegraphics[width=\textwidth]{images/TrafficTwoMonth/data_transformation/Aggr/traffic_linear_regression_evaluation_No Aggregation.png}
    \caption{Linear Regression Evaluation – No Aggregation}
\end{subfigure}
\hfill
\begin{subfigure}[b]{0.40\textwidth}
    \centering
    \includegraphics[width=\textwidth]{images/TrafficTwoMonth/data_transformation/Aggr/traffic_persistence_optim_eval_No Aggregation.png}
    \caption{Persistence Optim Evaluation – No Aggregation}
\end{subfigure}

\vspace{1em}

\begin{subfigure}[b]{0.40\textwidth}
    \centering
    \includegraphics[width=\textwidth]{images/TrafficTwoMonth/data_transformation/Aggr/traffic_linear_regression_evaluation_30min Aggregation.png}
    \caption{Linear Regression Evaluation – 30min Aggregation}
\end{subfigure}
\hfill
\begin{subfigure}[b]{0.40\textwidth}
    \centering
    \includegraphics[width=\textwidth]{images/TrafficTwoMonth/data_transformation/Aggr/traffic_persistence_optim_eval_30min Aggregation.png}
    \caption{Persistence Optim Evaluation – 30min Aggregation}
\end{subfigure}

\vspace{1em}

\begin{subfigure}[b]{0.40\textwidth}
    \centering
    \includegraphics[width=\textwidth]{images/TrafficTwoMonth/data_transformation/Aggr/traffic_linear_regression_evaluation_Hourly Aggregation.png}
    \caption{Linear Regression Evaluation – Hourly Aggregation}
\end{subfigure}
\hfill
\begin{subfigure}[b]{0.40\textwidth}
    \centering
    \includegraphics[width=\textwidth]{images/TrafficTwoMonth/data_transformation/Aggr/traffic_persistence_optim_eval_Hourly Aggregation.png}
    \caption{Persistence Optim Evaluation – Hourly Aggregation}
\end{subfigure}

\vspace{1em}

\begin{subfigure}[b]{0.40\textwidth}
    \centering
    \includegraphics[width=\textwidth]{images/TrafficTwoMonth/data_transformation/Aggr/traffic_linear_regression_evaluation_Daily Aggregation.png}
    \caption{Linear Regression Evaluation – Daily Aggregation}
\end{subfigure}
\hfill
\begin{subfigure}[b]{0.40\textwidth}
    \centering
    \includegraphics[width=\textwidth]{images/TrafficTwoMonth/data_transformation/Aggr/traffic_persistence_optim_eval_Daily Aggregation.png}
    \caption{Persistence Optim Evaluation – Daily Aggregation}
\end{subfigure}

\vspace{1em}

\begin{subfigure}[b]{0.40\textwidth}
    \centering
    \includegraphics[width=\textwidth]{images/TrafficTwoMonth/data_transformation/Aggr/traffic_linear_regression_evaluation_Weekly Aggregation.png}
    \caption{Linear Regression Evaluation – Weekly Aggregation}
\end{subfigure}
\hfill
\begin{subfigure}[b]{0.40\textwidth}
    \centering
    \includegraphics[width=\textwidth]{images/TrafficTwoMonth/data_transformation/Aggr/traffic_persistence_optim_eval_Weekly Aggregation.png}
    \caption{Persistence Optim Evaluation – Weekly Aggregation}
\end{subfigure}

\caption{Evaluation results for Linear Regression and Persistence Optim after different aggregations}
\label{fig:evaluation_results}
\end{figure}

\subsection*{\textit{Smoothing}}
We tested moving-average smoothing with window sizes of 3, 5, 10, and 15 to see how noise reduction affects forecasting without distorting the data.Smaller windows kept more noise, while larger ones over-smoothed.The window of 15 worked best, giving RMSE ~0.3, MAE ~0.25, and R² ~0.96 in the optimistic persistence model.Linear regression stayed flat with low R², so we chose size 15.

\begin{figure}[H]
\centering
\begin{subfigure}[b]{0.45\textwidth}
    \centering
    \includegraphics[width=\textwidth]{images/TrafficTwoMonth/data_transformation/Smooth/traffic_smoothed_size_3.png}
    \caption{Smoothing Size 3}
    \label{fig:smooth:3}
\end{subfigure}
\hfill
\begin{subfigure}[b]{0.45\textwidth}
    \centering
    \includegraphics[width=\textwidth]{images/TrafficTwoMonth/data_transformation/Smooth/traffic_smoothed_size_5.png}
    \caption{Smoothing Size 5}
    \label{fig:smooth:5}
\end{subfigure}

\vspace{1em}

\begin{subfigure}[b]{0.45\textwidth}
    \centering
    \includegraphics[width=\textwidth]{images/TrafficTwoMonth/data_transformation/Smooth/traffic_smoothed_size_10.png}
    \caption{Smoothing Size 10}
    \label{fig:smooth:10}
\end{subfigure}
\hfill
\begin{subfigure}[b]{0.45\textwidth}
    \centering
    \includegraphics[width=\textwidth]{images/TrafficTwoMonth/data_transformation/Smooth/traffic_smoothed_size_15.png}
    \caption{Smoothing Size 15}
    \label{fig:smooth:15}
\end{subfigure}

\caption{Time series plots after applying moving average smoothing with different window sizes}
\label{fig:smoothing_plots}
\end{figure}

\begin{figure}[H]
\centering
\begin{subfigure}[b]{0.45\textwidth}
    \centering
    \includegraphics[width=\textwidth]{images/TrafficTwoMonth/data_transformation/Smooth/traffic_linear_regression_forecast_Smoothing Size 3.png}
    \caption{Linear Regression – Size 3}
\end{subfigure}
\hfill
\begin{subfigure}[b]{0.45\textwidth}
    \centering
    \includegraphics[width=\textwidth]{images/TrafficTwoMonth/data_transformation/Smooth/traffic_persistence_optim_forecast__Smoothing Size 3.png}
    \caption{Persistence Optim – Size 3}
\end{subfigure}

\vspace{1em}

\begin{subfigure}[b]{0.45\textwidth}
    \centering
    \includegraphics[width=\textwidth]{images/TrafficTwoMonth/data_transformation/Smooth/traffic_linear_regression_forecast_Smoothing Size 5.png}
    \caption{Linear Regression – Size 5}
\end{subfigure}
\hfill
\begin{subfigure}[b]{0.45\textwidth}
    \centering
    \includegraphics[width=\textwidth]{images/TrafficTwoMonth/data_transformation/Smooth/traffic_persistence_optim_forecast__Smoothing Size 5.png}
    \caption{Persistence Optim – Size 5}
\end{subfigure}

\vspace{1em}

\begin{subfigure}[b]{0.45\textwidth}
    \centering
    \includegraphics[width=\textwidth]{images/TrafficTwoMonth/data_transformation/Smooth/traffic_linear_regression_forecast_Smoothing Size 10.png}
    \caption{Linear Regression – Size 10}
\end{subfigure}
\hfill
\begin{subfigure}[b]{0.45\textwidth}
    \centering
    \includegraphics[width=\textwidth]{images/TrafficTwoMonth/data_transformation/Smooth/traffic_persistence_optim_forecast__Smoothing Size 10.png}
    \caption{Persistence Optim – Size 10}
\end{subfigure}

\vspace{1em}

\begin{subfigure}[b]{0.45\textwidth}
    \centering
    \includegraphics[width=\textwidth]{images/TrafficTwoMonth/data_transformation/Smooth/traffic_linear_regression_forecast_Smoothing Size 15.png}
    \caption{Linear Regression – Size 15}
\end{subfigure}
\hfill
\begin{subfigure}[b]{0.45\textwidth}
    \centering
    \includegraphics[width=\textwidth]{images/TrafficTwoMonth/data_transformation/Smooth/traffic_persistence_optim_forecast__Smoothing Size 15.png}
    \caption{Persistence Optim – Size 15}
\end{subfigure}

\caption{Forecasting plots for Linear Regression and Persistence Optim after different smoothing window sizes}
\label{fig:smoothing_forecasts}
\end{figure}

\begin{figure}[H]
\centering
\begin{subfigure}[b]{0.45\textwidth}
    \centering
    \includegraphics[width=\textwidth]{images/TrafficTwoMonth/data_transformation/Smooth/traffic_linear_regression_evaluation_Smoothing Size 3.png}
    \caption{Linear Regression Evaluation – Size 3}
\end{subfigure}
\hfill
\begin{subfigure}[b]{0.45\textwidth}
    \centering
    \includegraphics[width=\textwidth]{images/TrafficTwoMonth/data_transformation/Smooth/traffic_persistence_optim_eval_Smoothing Size 3.png}
    \caption{Persistence Optim Evaluation – Size 3}
\end{subfigure}

\vspace{1em}

\begin{subfigure}[b]{0.45\textwidth}
    \centering
    \includegraphics[width=\textwidth]{images/TrafficTwoMonth/data_transformation/Smooth/traffic_linear_regression_evaluation_Smoothing Size 5.png}
    \caption{Linear Regression Evaluation – Size 5}
\end{subfigure}
\hfill
\begin{subfigure}[b]{0.45\textwidth}
    \centering
    \includegraphics[width=\textwidth]{images/TrafficTwoMonth/data_transformation/Smooth/traffic_persistence_optim_eval_Smoothing Size 5.png}
    \caption{Persistence Optim Evaluation – Size 5}
\end{subfigure}

\vspace{1em}

\begin{subfigure}[b]{0.45\textwidth}
    \centering
    \includegraphics[width=\textwidth]{images/TrafficTwoMonth/data_transformation/Smooth/traffic_linear_regression_evaluation_Smoothing Size 10.png}
    \caption{Linear Regression Evaluation – Size 10}
\end{subfigure}
\hfill
\begin{subfigure}[b]{0.45\textwidth}
    \centering
    \includegraphics[width=\textwidth]{images/TrafficTwoMonth/data_transformation/Smooth/traffic_persistence_optim_eval_Smoothing Size 10.png}
    \caption{Persistence Optim Evaluation – Size 10}
\end{subfigure}

\vspace{1em}

\begin{subfigure}[b]{0.45\textwidth}
    \centering
    \includegraphics[width=\textwidth]{images/TrafficTwoMonth/data_transformation/Smooth/traffic_linear_regression_evaluation_Smoothing Size 15.png}
    \caption{Linear Regression Evaluation – Size 15}
\end{subfigure}
\hfill
\begin{subfigure}[b]{0.45\textwidth}
    \centering
    \includegraphics[width=\textwidth]{images/TrafficTwoMonth/data_transformation/Smooth/traffic_persistence_optim_eval_Smoothing Size 15.png}
    \caption{Persistence Optim Evaluation – Size 15}
\end{subfigure}

\caption{Evaluation results for Linear Regression and Persistence Optim after different smoothing window sizes}
\label{fig:smoothing_evaluations}
\end{figure}

\subsection*{\textit{Differentiation}}
To handle possible non‑stationarity, we tested no differencing, first‑order, and second‑order differencing.No differencing kept the trend, first‑order removed linear trends and centered the data, and second‑order added noise without helping.The optimistic persistence model performed best with no differencing, reaching R² ~0.95.Linear regression improved only slightly, so we chose no differencing.

\begin{figure}[H]
\centering
\begin{subfigure}[b]{0.45\textwidth}
    \centering
    \includegraphics[width=\textwidth]{images/TrafficTwoMonth/data_transformation/Diff/traffic_no_diff.png}
    \caption{No Differentiation}
    \label{fig:diff:no}
\end{subfigure}
\hfill
\begin{subfigure}[b]{0.45\textwidth}
    \centering
    \includegraphics[width=\textwidth]{images/TrafficTwoMonth/data_transformation/Diff/traffic_first_diff.png}
    \caption{First Differentiation}
    \label{fig:diff:first}
\end{subfigure}

\vspace{1em}

\begin{subfigure}[b]{0.45\textwidth}
    \centering
    \includegraphics[width=\textwidth]{images/TrafficTwoMonth/data_transformation/Diff/traffic_second_diff.png}
    \caption{Second Differentiation}
    \label{fig:diff:second}
\end{subfigure}

\caption{Time series plots without and after applying first, and second differentiation}
\label{fig:differentiation_plots}
\end{figure}

\begin{figure}[H]
\centering
\begin{subfigure}[b]{0.45\textwidth}
    \centering
    \includegraphics[width=\textwidth]{images/TrafficTwoMonth/data_transformation/Diff/traffic_linear_regression_forecast_No_Differentiation.png}
    \caption{Linear Regression – No Differentiation}
\end{subfigure}
\hfill
\begin{subfigure}[b]{0.45\textwidth}
    \centering
    \includegraphics[width=\textwidth]{images/TrafficTwoMonth/data_transformation/Diff/traffic_persistence_optim_forecast__No Differentiation.png}
    \caption{Persistence Optim – No Differentiation}
\end{subfigure}

\vspace{1em}

\begin{subfigure}[b]{0.45\textwidth}
    \centering
    \includegraphics[width=\textwidth]{images/TrafficTwoMonth/data_transformation/Diff/traffic_linear_regression_forecast_First Differentiation.png}
    \caption{Linear Regression – First Differentiation}
\end{subfigure}
\hfill
\begin{subfigure}[b]{0.45\textwidth}
    \centering
    \includegraphics[width=\textwidth]{images/TrafficTwoMonth/data_transformation/Diff/traffic_persistence_optim_forecast__First Differentiation.png}
    \caption{Persistence Optim – First Differentiation}
\end{subfigure}

\vspace{1em}

\begin{subfigure}[b]{0.45\textwidth}
    \centering
    \includegraphics[width=\textwidth]{images/TrafficTwoMonth/data_transformation/Diff/traffic_linear_regression_forecast_Second Differentiation.png}
    \caption{Linear Regression – Second Differentiation}
\end{subfigure}
\hfill
\begin{subfigure}[b]{0.45\textwidth}
    \centering
    \includegraphics[width=\textwidth]{images/TrafficTwoMonth/data_transformation/Diff/traffic_persistence_optim_forecast__Second Differentiation.png}
    \caption{Persistence Optim – Second Differentiation}
\end{subfigure}

\caption{Forecasting plots for Linear Regression and Persistence Optim after different levels of differentiation}
\label{fig:differentiation_forecasts}
\end{figure}

\begin{figure}[H]
\centering
\begin{subfigure}[b]{0.45\textwidth}
    \centering
    \includegraphics[width=\textwidth]{images/TrafficTwoMonth/data_transformation/Diff/traffic_linear_regression_evaluation_No Differentiation.png}
    \caption{Linear Regression Evaluation – No Differentiation}
\end{subfigure}
\hfill
\begin{subfigure}[b]{0.45\textwidth}
    \centering
    \includegraphics[width=\textwidth]{images/TrafficTwoMonth/data_transformation/Diff/traffic_persistence_optim_eval_No Differentiation.png}
    \caption{Persistence Optim Evaluation – No Differentiation}
\end{subfigure}

\vspace{1em}

\begin{subfigure}[b]{0.45\textwidth}
    \centering
    \includegraphics[width=\textwidth]{images/TrafficTwoMonth/data_transformation/Diff/traffic_linear_regression_evaluation_First Differentiation.png}
    \caption{Linear Regression Evaluation – First Differentiation}
\end{subfigure}
\hfill
\begin{subfigure}[b]{0.45\textwidth}
    \centering
    \includegraphics[width=\textwidth]{images/TrafficTwoMonth/data_transformation/Diff/traffic_persistence_optim_eval_First Differentiation.png}
    \caption{Persistence Optim Evaluation – First Differentiation}
\end{subfigure}

\vspace{1em}

\begin{subfigure}[b]{0.45\textwidth}
    \centering
    \includegraphics[width=\textwidth]{images/TrafficTwoMonth/data_transformation/Diff/traffic_linear_regression_evaluation_Second Differentiation.png}
    \caption{Linear Regression Evaluation – Second Differentiation}
\end{subfigure}
\hfill
\begin{subfigure}[b]{0.45\textwidth}
    \centering
    \includegraphics[width=\textwidth]{images/TrafficTwoMonth/data_transformation/Diff/traffic_persistence_optim_eval_Second Differentiation.png}
    \caption{Persistence Optim Evaluation – Second Differentiation}
\end{subfigure}

\caption{Evaluation results for Linear Regression and Persistence Optim after different levels of differentiation}
\label{fig:differentiation_evaluations}
\end{figure}

\subsection*{\textit{Scaling}}
We used StandardScaler to normalize the data, converting it to a mean of 0 and a standard deviation of 1.Before scaling, the values ranged from about 25 to 250 with high variance; after scaling, they fell roughly between –1 and 3.This improved model stability,and sped up convergence as shown in the before‑and‑after plots.We did not test other scaling methods because StandardScaler solved the scale issues without adding complexity.The final preparation pipeline for the traffic dataset is: 30‑minute aggregation, smoothing with window size 15, no differencing, and scaling with StandardScaler.

\begin{figure}[H]
\centering
\begin{subfigure}{0.8\textwidth}
    \centering
    \includegraphics[width=\textwidth]{images/TrafficTwoMonth/data_transformation/Scale/Traffic_before_scaling.png}
    \caption{Before scaling}
\end{subfigure}
\par\bigskip
\begin{subfigure}{0.8\textwidth}
    \centering
    \includegraphics[width=\textwidth]{images/TrafficTwoMonth/data_transformation/Scale/Traffic_after_scaling.png}
    \caption{After StandardScaler()}
\end{subfigure}
\caption{Effect of StandardScaler on the original 15-minute time series}
\end{figure}

\section{MODELS' EVALUATION}

\subsection*{Exponential Smoothing Model}

We applied simple exponential smoothing and tuned alpha from 0.1 to 0.9. The hyperparameter study (Figure~\ref{fig:exp_study}) shows R² starts near 0.0 at alpha = 0.1 and drops sharply to ~–1.0 by alpha = 0.3, indicating the model overreacts to recent changes.

The forecast (Figure \ref{fig:exp_forecast}) follows the overall trend but smooths out sharp variations. The evaluation (Figure~\ref{fig:exp_eval}) shows poor test performance: RMSE $\approx$ 0.24, MAE $\approx$ 0.20, MAPE $\approx$ 100\%, R² $\approx$ --0.01, revealing weak accuracy and poor generalization.


\begin{figure}[H]
\centering
\includegraphics[width=0.5\textwidth]{images/TrafficTwoMonth/models_evalution/exponcialSmoothing/traffic_exponential_smoothing_R2_study_30.png}
\caption{Hyperparameter study: R² as a function of alpha for Exponential Smoothing}
\label{fig:exp_study}
\end{figure}

\begin{figure}[H]
\centering
\includegraphics[width=0.9\textwidth]{images/TrafficTwoMonth/models_evalution/exponcialSmoothing/traffic_exponential_smoothing_R2_forecast_30.png}
\caption{Forecasting plots obtained with the best Exponential Smoothing model (predictions in red vs actual test data in pink)}
\label{fig:exp_forecast}
\end{figure}

\begin{figure}[H]
\centering
\includegraphics[width=0.8\textwidth]{images/TrafficTwoMonth/models_evalution/exponcialSmoothing/traffic_exponential_smoothing_R2_eval_30.png}
\caption{Performance metrics of the best Exponential Smoothing model (RMSE, MAE, MAPE, R²)}
\label{fig:exp_eval}
\end{figure}

\subsection*{Multi-layer Perceptrons Model}

For the MLP model, we explored various hidden layer architectures, including single layers with 50 or 100 neurons, and multi-layers such as (50, 50) and (100, 50), trained with the Adam optimizer and MSE loss.The best setup was (50, 50), reaching test R² $\approx 0.86$ after $\sim$800 epochs with stable training and no overfitting.The forecasting plot (Figure~\ref{fig:mlp_forecast}) closely matched the test data, capturing nonlinear patterns better than simpler models.Performance metrics (Figure~\ref{fig:mlp_eval}) show RMSE $\approx 0.25$, MAE $\approx 0.20$, MAPE $\approx 18\%$, and R² $\approx 0.86$, confirming solid medium-term performance despite higher computational cost.

\begin{figure}[H]
\centering
\includegraphics[width=0.8\textwidth]{images/TrafficTwoMonth/models_evalution/Multi-layerPerceptrons/traffic_mlp_R2_study.png}
\caption{Hyperparameter study: R² convergence for different MLP hidden layer configurations}
\label{fig:mlp_study}
\end{figure}

\begin{figure}[H]
\centering
\includegraphics[width=0.9\textwidth]{images/TrafficTwoMonth/models_evalution/Multi-layerPerceptrons/traffic_mlp_R2_forecast.png}
\caption{Forecasting plots obtained with the best MLP model (predictions in red vs actual test data in pink)}
\label{fig:mlp_forecast}
\end{figure}

\begin{figure}[H]
\centering
\includegraphics[width=0.8\textwidth]{images/TrafficTwoMonth/models_evalution/Multi-layerPerceptrons/traffic_mlp_R2_eval.png}
\caption{Performance metrics of the best MLP model (RMSE, MAE, MAPE, R²)}
\label{fig:mlp_eval}
\end{figure}

\subsection*{ARIMA Model}

We ran a grid search for ARIMA with p and q from 0 to 8 and d from 0 to 2.The study (Figure \ref{fig:arima_study}) highlights the optimal univariate configuration as p=5, d=1, q=7.ARIMA fit training data well (R² $\approx$ 0.85) but dropped on test (R² $\approx$ 0.08), with RMSE rising from 0.08 to 0.21 (Figure \ref{fig:arima_eval}). VAR with lag=4 showed more stable training (R² $\approx$ 0.82) but also weak test performance (R² $\approx$ 0.06) (Figure \ref{fig:var_eval}).Both models captured trends but failed on sharp changes, with high MAPE (~100\%) and limited generalization. (Figure \ref{fig:arima_forecast})
\begin{figure}[H]
\centering
\includegraphics[width=0.7\textwidth]{images/TrafficTwoMonth/models_evalution/ARIMA/traffic_arima_R2_study.png}
\caption{Hyperparameter study: best ARIMA configuration (p=5, d=1, q=7) – univariate}
\label{fig:arima_study}
\end{figure}

\begin{figure}[H]
\centering
\includegraphics[width=0.9\textwidth]{images/TrafficTwoMonth/models_evalution/ARIMA/traffic_arima_R2_forecast_30.png}
\caption{Forecasting plots obtained with the best ARIMA model (predictions in red vs actual test data in pink) – univariate}
\label{fig:arima_forecast}
\end{figure}

\begin{figure}[H]
\centering
\includegraphics[width=0.8\textwidth]{images/TrafficTwoMonth/models_evalution/ARIMA/traffic_arima_R2_eval_30.png}
\caption{Performance metrics of the best ARIMA model (RMSE, MAE, MAPE, R²) – univariate}
\label{fig:arima_eval}
\end{figure}

\begin{figure}[H]
\centering
\includegraphics[width=0.7\textwidth]{images/TrafficTwoMonth/models_evalution/ARIMA/traffic_30min_first_diff_arima_R2_study_multi (1).png}
\caption{Hyperparameter study: best VAR configuration (lag=4) – multivariate}
\label{fig:var_study}
\end{figure}

\begin{figure}[H]
\centering
\includegraphics[width=0.9\textwidth]{images/TrafficTwoMonth/models_evalution/ARIMA/traffic_30min_first_diff_var_forecast_30.png}
\caption{Forecasting plots obtained with the best VAR model (predictions in red vs actual test data in pink) – multivariate}
\label{fig:var_forecast}
\end{figure}

\begin{figure}[H]
\centering
\includegraphics[width=0.8\textwidth]{images/TrafficTwoMonth/models_evalution/ARIMA/traffic_30min_first_diff_var_eval_30.png}
\caption{Performance metrics of the best VAR model (RMSE, MAE, MAPE, R²) – multivariate}
\label{fig:var_eval}

\end{figure}

\subsection*{LSTMs Model}

We tuned LSTM with sequence lengths 3-6, hidden units 20-50, and up to 2500 epochs using Adam and MSE. The study (Figure \ref{fig:lstm_study}) identifies the best univariate setup was seq=4, hidden=25, epochs=2100, reaching test R² ~0.91.The forecasting plot (Figure \ref{fig:lstm_forecast}) closely matched test data, capturing trends and irregularities.Metrics (Figure \ref{fig:lstm_eval}) report test RMSE ~0.20, MAE ~0.15, MAPE ~13\%, R² 0.84.In the multivariate case, similar tuning led to stable R² ~0.87 and better generalization across variables.

\begin{figure}[H]
\centering
\includegraphics[width=0.7\textwidth]{images/TrafficTwoMonth/models_evalution/LSTM/Screenshot 2025-12-18 214052.png}
\caption{Hyperparameter study: best LSTM configuration (sequence length=4, hidden=25, epochs=2100) – univariate}
\label{fig:lstm_study}
\end{figure}

\begin{figure}[H]
\centering
\includegraphics[width=0.9\textwidth]{images/TrafficTwoMonth/models_evalution/LSTM/traffic_lstms_R2_forecast_15.png}
\caption{Forecasting plots obtained with the best LSTM model (predictions in red vs actual test data in pink) – univariate}
\label{fig:lstm_forecast}
\end{figure}

\begin{figure}[H]
\centering
\includegraphics[width=0.8\textwidth]{images/TrafficTwoMonth/models_evalution/LSTM/traffic_lstms_R2_eval (1).png}
\caption{Performance metrics of the best LSTM model (RMSE, MAE, MAPE, R²) – univariate}
\label{fig:lstm_eval}
\end{figure}

\begin{figure}[H]
\centering
\includegraphics[width=0.7\textwidth]{images/TrafficTwoMonth/models_evalution/LSTM/Screenshot 2026-01-03 162427.png}
\caption{Hyperparameter study: best LSTM configuration – multivariate}
\label{fig:lstm_study_ml}
\end{figure}

\begin{figure}[H]
\centering
\includegraphics[width=0.9\textwidth]{images/TrafficTwoMonth/models_evalution/LSTM/traffic_30min_first_diff_lstm_multivariate_r2_forecast_15 (1).png}
\caption{Forecasting plots obtained with the best LSTM model (predictions in red vs actual test data in pink) – multivariate}
\label{fig:lstm_forecast_ml}
\end{figure}

\begin{figure}[H]
\centering
\includegraphics[width=0.8\textwidth]{images/TrafficTwoMonth/models_evalution/LSTM/traffic_30min_first_diff_lstm_multivariate_r2_eval (1).png}
\caption{Performance metrics of the best LSTM model (RMSE, MAE, MAPE, R²) – multivariate}
\label{fig:lstm_eval_ml}
\end{figure}


\section{CRITICAL ANALYSIS}
The forecasting models showed distinct performance differences on the processed traffic data. Exponential smoothing performed the worst, with test R² near zero, very high MAPE, and poor ability to follow variability, making it unsuitable for this task.ARIMA and VAR fit the training data well but generalized poorly, with low test R² ($\approx$ 0.08 and 0.06). Both models captured overall trends but failed on sharp fluctuations due to their linear assumptions and limited flexibility.MLP delivered stronger results, reaching R² $\approx$ 0.86 and capturing nonlinear patterns more effectively, though at a higher computational cost. LSTM clearly outperformed all other models.The univariate version reached R² $\approx$ 0.84, while the multivariate version achieved $\approx$ 0.87, with lower RMSE and MAPE.LSTM handled sequential dependencies, irregularities, and local variations better than statistical models and even better than MLP, making it the most reliable option for medium-term traffic forecasting.
Aggregating the data into 30-minute intervals reduced high-frequency noise and exposed clearer temporal patterns. Smoothing with a window of 15 further stabilized the series without removing essential structure.Avoiding differencing prevented unnecessary noise amplification, since the aggregated and smoothed series was already close to stationary.Scaling was essential for neural models, ensuring stable gradients and faster convergence.Together, these steps transformed noisy raw data into a predictable signal that models could learn effectively.

\end{document}