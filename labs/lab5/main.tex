\documentclass[10pt]{extarticle}
\renewcommand{\arraystretch}{1.5}
\renewcommand{\baselinestretch}{1.5}
\usepackage[onehalfspacing]{setspace}
\setlength{\parindent}{0em}
\setlength{\parskip}{0.2em}
\font\myfont=cmr12 at 26pt
\usepackage{anyfontsize}
\usepackage{tabularx}
\usepackage{multirow}
\pagenumbering{arabic} 
\usepackage{soul}
\usepackage{xcolor}
\usepackage[T1]{fontenc}
\renewcommand{\familydefault}{\sfdefault}
\usepackage{blindtext}
\usepackage{titling}
\setlength{\droptitle}{-14em}   % This is your set screw
\usepackage[english]{babel}
\usepackage{graphicx}
\usepackage{float}
\usepackage{eso-pic}
\graphicspath{ {./images/} }
\usepackage{subcaption}
\usepackage{geometry}
\usepackage[section]{placeins}
\geometry{margin=2cm, bmargin=2cm, tmargin=3cm}

\newcommand{\ctext}[3][RGB]{%
  \begingroup
  \definecolor{hlcolor}{#1}{#2}\sethlcolor{hlcolor}%
  \hl{#3}%
  \endgroup
}


\begin{document}


\AddToShipoutPictureBG*{
\AtPageUpperLeft{
\hspace{19.5cm}
\raisebox{-2.5cm}{\makebox[0pt][r]{\fontsize{36}{1cm}\selectfont DS 2024\par}}}}

\AddToShipoutPictureBG*{
\AtPageUpperLeft{
\hspace{6.5cm}
\raisebox{-3.5cm}{
\makebox[0pt][r]{ 
\includegraphics[scale=0.95]{tecnico_logo.jpg}\\[3cm]}}}}


\title{{\myfont Data Science Project}}  % Title
\setlength{\droptitle}{1cm}

\date{\vspace{-9ex}} % Date for the report, skipped and used to adjust height
\maketitle % Insert the title, author and date
\begin{center}
    %\setlength\extrarowheight{7pt}
    \begin{tabular}{ |l|l l|l| }
        \hline
        \multirow{4}{6em}{\textbf{Team nr: } 16} & \textbf{Student 1: } Antero Morgado & \textbf{IST nr: } 1119213\\
        & \textbf{Student 2: } David Ferreira & \textbf{IST nr:} 1107077 \\
        & \textbf{Student 3: } José Fernandes & \textbf{IST nr: } 1103727 \\
        & \textbf{Student 4: } Olha Buts & \textbf{IST nr: } 1116276 \\
        \hline
    \end{tabular}
\end{center}

\begin{center}
	\section*{\fontsize{0.75cm}{1cm}\selectfont TIME SERIES ANALYSIS}
\end{center}

\section{DATA PROFILING}

\subsection*{\textit{Data Dimensionality and Granularity}}
\ctext[RGB]{190,190,190}{May be used to identify the most atomic granularity and two other different granularities to consider.  \textbf{Shall not exceed 500 characters.}}

\begin{figure}[H]
%\centering\includegraphics[scale=0.95]{}
\caption{Time series 1 at the most granular detail}
\end{figure}

\begin{figure}[H]
%\centering\includegraphics[scale=0.95]{}
\caption{Time series 1 at the second chosen granularity}
\end{figure}

\begin{figure}[H]
%\centering\includegraphics[scale=0.95]{}
\caption{Time series 1 at the third chosen granularity}
\end{figure}

\begin{figure}[H]
%\centering\includegraphics[scale=0.95]{}
\caption{Time series 2 at the most granular detail}
\end{figure}

\begin{figure}[H]
%\centering\includegraphics[scale=0.95]{}
\caption{Time series 2 at the second chosen granularity}
\end{figure}

\begin{figure}[H]
%\centering\includegraphics[scale=0.95]{}
\caption{Time series 2 at the third chosen granularity}
\end{figure}

\subsection*{\textit{Data Distribution}}
\ctext[RGB]{190,190,190}{Shall be used to perform the data analysis at those three different granularities, concerning the series distribution.  \textbf{Shall not exceed 500 characters.}}

\begin{figure}[H]
%\centering\includegraphics[scale=0.95]{}
\caption{Boxplot(s) for time series 1}
\end{figure}

\begin{figure}[H]
%\centering\includegraphics[scale=0.95]{}
\caption{Boxplot(s) for time series 2}
\end{figure}

\begin{figure}[H]
%\centering\includegraphics[scale=0.95]{}
\caption{Histogram(s) for time series 1}
\end{figure}

\begin{figure}[H]
%\centering\includegraphics[scale=0.95]{}
\caption{Histogram(s) for time series 2}
\end{figure}

\begin{figure}[H]
%\centering\includegraphics[scale=0.95]{}
\caption{Autocorrelation lag-plots for original time series 1}
\end{figure}

\begin{figure}[H]
%\centering\includegraphics[scale=0.95]{}
\caption{Autocorrelation lag-plots for original time series 2}
\end{figure}

\begin{figure}[H]
%\centering\includegraphics[scale=0.95]{}
\caption{Autocorrelation correlogram for original time series 1}
\end{figure}

\begin{figure}[H]
%\centering\includegraphics[scale=0.95]{}
\caption{Autocorrelation correlogram for original time series 2}
\end{figure}

\subsection*{\textit{Data Stationarity}}
\ctext[RGB]{190,190,190}{Shall be used to perform the data analysis at those three different granularities, concerning the series stationarity.  \textbf{Shall not exceed 300 characters.}}

\begin{figure}[H]
%\centering\includegraphics[scale=0.95]{}
\caption{Components study for time series 1}
\end{figure}


\begin{figure}[H]
%\centering\includegraphics[scale=0.95]{}
\caption{Stationarity study for time series 1}
\end{figure}

\begin{figure}[H]
%\centering\includegraphics[scale=0.95]{}
\caption{Components study for time series 2}
\end{figure}


\begin{figure}[H]
%\centering\includegraphics[scale=0.95]{}
\caption{Stationarity study for time series 2}
\end{figure}

\end{document}