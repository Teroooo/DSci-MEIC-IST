\documentclass[10pt]{extarticle}
\renewcommand{\arraystretch}{1.5}
\renewcommand{\baselinestretch}{1.5}
\usepackage[onehalfspacing]{setspace}
\setlength{\parindent}{0em}
\setlength{\parskip}{0.2em}
\font\myfont=cmr12 at 26pt
\usepackage{anyfontsize}
\usepackage{tabularx}
\usepackage{multirow}
\pagenumbering{arabic} 
\usepackage{soul}
\usepackage{xcolor}
\usepackage[T1]{fontenc}
\renewcommand{\familydefault}{\sfdefault}
\usepackage{blindtext}
\usepackage{titling}
\setlength{\droptitle}{-14em}   % This is your set screw
\usepackage[english]{babel}
\usepackage{graphicx}
\usepackage{float}
\usepackage{eso-pic}
\graphicspath{ {./images/} }
\usepackage{subcaption}
\usepackage{geometry}
\usepackage[section]{placeins}
\geometry{margin=2cm, bmargin=2cm, tmargin=3cm}

\newcommand{\ctext}[3][RGB]{%
  \begingroup
  \definecolor{hlcolor}{#1}{#2}\sethlcolor{hlcolor}%
  \hl{#3}%
  \endgroup
}


\begin{document}


\AddToShipoutPictureBG*{
\AtPageUpperLeft{
\hspace{19.5cm}
\raisebox{-2.5cm}{\makebox[0pt][r]{\fontsize{36}{1cm}\selectfont DS 2024\par}}}}

\AddToShipoutPictureBG*{
\AtPageUpperLeft{
\hspace{6.5cm}
\raisebox{-3.5cm}{
\makebox[0pt][r]{ 
\includegraphics[scale=0.95]{tecnico_logo.jpg}\\[3cm]}}}}


\title{{\myfont Data Science Lab1}}  % Title
\setlength{\droptitle}{1cm}

\date{\vspace{-9ex}} % Date for the report, skipped and used to adjust height
\maketitle % Insert the title, author and date
\begin{center}
    %\setlength\extrarowheight{7pt}
    \begin{tabular}{ |l|l l|l| }
        \hline
        \multirow{4}{6em}{\textbf{Team nr: } Insert here} & \textbf{Student 1: } Antero Morgado & \textbf{IST nr: } 1119213\\
        & \textbf{Student 2: } David Ferreira & \textbf{IST nr:} 1107077 \\
        & \textbf{Student 3: } José Fernandes & \textbf{IST nr: } 1103727 \\
        & \textbf{Student 4: } Olha Buts & \textbf{IST nr: } 1116276 \\
        \hline
    \end{tabular}
\end{center}

\vspace{2cm}

\begin{center}
	\section*{\fontsize{0.75cm}{1cm}\selectfont CLASSIFICATION}
\end{center}

\section{DATA PROFILING}
\ctext[RGB]{190,190,190}{May be used to describe any useful observation about the data, and that was used in the current project. An example is the use of any domain knowledge to process the data or evaluate the results. \textbf{Shall not exceed 200 characters.}}

\subsection*{\textit{Data Dimensionality}}
\ctext[RGB]{190,190,190}{Shall contain all relevant information and charts respecting to the data dimensionality perspective, such as the number of records and number of dimensions, and their impact on the following analysis. \textbf{Shall not exceed 500 characters.}}

\begin{figure}[H]
%\centering\includegraphics[scale=0.95]{}
\caption{Nr Records x Nr variables for dataset 1 (left) and dataset 2 (right)}
\end{figure}

\begin{figure}[H]
%\centering\includegraphics[scale=0.95]{}
\caption{Nr variables per type for dataset 1 (left) and dataset 2 (right)}
\end{figure}

\begin{figure}[H]
%\centering\includegraphics[scale=0.95]{}
\caption{Nr missing values for dataset 1 (left) and dataset 2 (right)}
\end{figure}

\subsection*{\textit{Data Distribution}}
\ctext[RGB]{190,190,190}{Shall contain all relevant information and charts respecting to the data distribution perspective, such as each variable distribution, type, domain and range. May be used to describe any useful observation about the data, and that was used in the current project.  \textbf{Shall not exceed 500 characters.}} 

\begin{figure}[H]
%\centering\includegraphics[scale=0.95]{}
\caption{Global boxplots dataset 1 (left) and dataset 2 (right)}
\end{figure}

\begin{figure}[H]
%\centering\includegraphics[scale=0.95]{}
\caption{Single variables boxplots for dataset 1}
\end{figure}

\begin{figure}[H]
%\centering\includegraphics[scale=0.95]{}
\caption{Single variables boxplots for dataset 2}
\end{figure}

\begin{figure}[H]
%\centering\includegraphics[scale=0.95]{}
\caption{Histograms for dataset 1} %(with distributions is enough)
\end{figure}

\begin{figure}[H]
%\centering\includegraphics[scale=0.95]{}
\caption{Histograms for dataset 2} %(with distributions is enough)
\end{figure}

\begin{figure}[H]
%\centering\includegraphics[scale=0.95]{}
\caption{Outliers study dataset 1}
\end{figure}

\begin{figure}[H]
%\centering\includegraphics[scale=0.95]{}
\caption{Outliers study dataset 2}
\end{figure}

\begin{figure}[H]
%\centering\includegraphics[scale=0.95]{}
\caption{Class distribution for dataset 1}
\end{figure}

\begin{figure}[H]
%\centering\includegraphics[scale=0.95]{}
\caption{Class distribution for dataset 2}
\end{figure}

\subsection*{\textit{Data Granularity}}
\ctext[RGB]{190,190,190}{Shall contain all relevant information and charts respecting to the data granularity perspective, such as the impact of different granularities considered for each variable. May present additional taxonomies if needed.  \textbf{Shall not exceed 500 characters.}}

\begin{figure}[H]
%\centering\includegraphics[scale=0.95]{}
\caption{Granularity analysis for dataset 1}
\end{figure}

\begin{figure}[H]
%\centering\includegraphics[scale=0.95]{}
\caption{Granularity analysis for dataset 2}
\end{figure}

\subsection*{\textit{Data Sparsity}}
\ctext[RGB]{190,190,190}{Shall contain all relevant information and charts respecting to the data sparsity perspective, such as domain coverage and correlation among variables.  \textbf{Shall not exceed 500 characters.}}

\begin{figure}[H]
%\centering\includegraphics[scale=0.95]{}
\caption{Sparsity analysis for dataset 1}
\end{figure}

\begin{figure}[H]
%\centering\includegraphics[scale=0.95]{}
\caption{Sparsity analysis for dataset 2}
\end{figure}

\begin{figure}[H]
%\centering\includegraphics[scale=0.95]{}
\caption{Correlation analysis for dataset 1}
\end{figure}

\begin{figure}[H]
%\centering\includegraphics[scale=0.95]{}
\caption{Correlation analysis for dataset 2}
\end{figure}

\end{document}